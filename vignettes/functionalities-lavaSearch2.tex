t% Created 2018-10-04 to 13:44
% Intended LaTeX compiler: pdflatex
\documentclass[12pt]{article}

%%%% settings when exporting code %%%% 

\usepackage{listings}
\lstset{
backgroundcolor=\color{white},
basewidth={0.5em,0.4em},
basicstyle=\ttfamily\small,
breakatwhitespace=false,
breaklines=true,
columns=fullflexible,
commentstyle=\color[rgb]{0.5,0,0.5},
frame=single,
keepspaces=true,
keywordstyle=\color{black},
literate={~}{$\sim$}{1},
numbers=left,
numbersep=10pt,
numberstyle=\ttfamily\tiny\color{gray},
showspaces=false,
showstringspaces=false,
stepnumber=1,
stringstyle=\color[rgb]{0,.5,0},
tabsize=4,
xleftmargin=.23in,
emph={anova,apply,class,coef,colnames,colNames,colSums,dim,dcast,for,ggplot,head,if,ifelse,is.na,lapply,list.files,library,logLik,melt,plot,require,rowSums,sapply,setcolorder,setkey,str,summary,tapply},
emphstyle=\color{blue}
}

%%%% packages %%%%%

\usepackage[utf8]{inputenc}
\usepackage[T1]{fontenc}
\usepackage{lmodern}
\usepackage{textcomp}
\usepackage{color}
\usepackage{enumerate}
\usepackage{graphicx}
\usepackage{grffile}
\usepackage{wrapfig}
\usepackage{rotating}
\usepackage{longtable}
\usepackage{multirow}
\usepackage{multicol}
\usepackage{changes}
\usepackage{pdflscape}
\usepackage{geometry}
\usepackage[normalem]{ulem}
\usepackage{amssymb}
\usepackage{amsmath}
\usepackage{amsfonts}
\usepackage{dsfont}
\usepackage{textcomp}
\usepackage{array}
\usepackage{ifthen}
\usepackage{hyperref}
\usepackage{natbib}
%\VignetteIndexEntry{Overview-lavaSearch2}
%\VignetteEngine{knitr::knitr}
\RequirePackage{fancyvrb}
\DefineVerbatimEnvironment{verbatim}{Verbatim}{fontsize=\small,formatcom = {\color[rgb]{0.5,0,0}}}
\geometry{a4paper, left=15mm, right=15mm}
\RequirePackage{colortbl} % arrayrulecolor to mix colors
\RequirePackage{setspace} % to modify the space between lines - incompatible with footnote in beamer
\usepackage{authblk} % enable several affiliations (clash with beamer)
\renewcommand{\baselinestretch}{1.1}
\geometry{top=1cm}
\usepackage{enumitem}
\RequirePackage{xspace} %
\newcommand\Rlogo{\textbf{\textsf{R}}\xspace} %
\RequirePackage{epstopdf} % to be able to convert .eps to .pdf image files
\author{Brice Ozenne}
\date{\today}
\title{Overview of the functionalities of the package lavaSearch2}
\hypersetup{
 colorlinks=true,
 citecolor=[rgb]{0,0.5,0},
 urlcolor=[rgb]{0,0,0.5},
 linkcolor=[rgb]{0,0,0.5},
 pdfauthor={Brice Ozenne},
 pdftitle={Overview of the functionalities of the package lavaSearch2},
 pdfkeywords={},
 pdfsubject={},
 pdfcreator={Emacs 25.2.1 (Org mode 9.0.4)},
 pdflang={English}
 }
\begin{document}

\maketitle
Load \textbf{lavaSearch2} in the R session:
\lstset{language=r,label= ,caption= ,captionpos=b,numbers=none}
\begin{lstlisting}
suppressPackageStartupMessages(library(lavaSearch2))
\end{lstlisting}

\section{Inference}
\label{sec:orgb5367e6}
\subsection{Introductory example}
\label{sec:orgd9de530}
You may have noticed that for simple linear regression, the p-values
of the Wald tests from \texttt{lm}:
\lstset{language=r,label= ,caption= ,captionpos=b,numbers=none}
\begin{lstlisting}
## simulate data
mSim <- lvm(Y[1:1]~0.3*X1+0.2*X2)
set.seed(10)
df.data <- sim(mSim, 2e1)

## fit linear model
summary(lm(Y~X1+X2, data = df.data))$coef
\end{lstlisting}

\begin{verbatim}
             Estimate Std. Error   t value    Pr(>|t|)
(Intercept) 0.7967775  0.2506767 3.1785069 0.005495832
X1          0.1550938  0.2205080 0.7033477 0.491360483
X2          0.4581556  0.2196785 2.0855736 0.052401103
\end{verbatim}

differ from those obtained with lava:
\lstset{language=r,label= ,caption= ,captionpos=b,numbers=none}
\begin{lstlisting}
## fit latent variable model
m <- lvm(Y~X1+X2)
e <- estimate(m, data = df.data)

## extract Wald tests
summary(e)$coef
\end{lstlisting}

\begin{verbatim}
      Estimate Std. Error   Z-value      P-value
Y~X1 0.1550938  0.2032984 0.7628877 0.4455303456
Y~X2 0.4581556  0.2025335 2.2621221 0.0236898575
Y~~Y 0.5557910  0.1757566 3.1622777           NA
Y    0.7967775  0.2311125 3.4475747 0.0005656439
\end{verbatim}

For instance, the p-value for the effect of X2 is 0.024 according to
lava and 0.052 according to \texttt{lm}. The discrepancy is due to 2
corrections that \texttt{lm} applies in order to improve the control of the
type 1 error of the Wald tests:
\begin{itemize}
\item use of a student distribution instead of a normal distribution
(informally t-value instead of z-value).
\item use of a bias-corrected estimator of the residuals variance instead
of the ML-estimator.
\end{itemize}
\textbf{lavaSearch2} attempts to generalize these corrections to models with
correlated and heteroschedastic measurements. In the case of a simple
linear regression, Wald tests obtained with \textbf{lavaSearch2} closely
approximate the results of \texttt{lm}:
\lstset{language=r,label= ,caption= ,captionpos=b,numbers=none}
\begin{lstlisting}
summary2(e)$coef
\end{lstlisting}

\begin{verbatim}
      Estimate Std. Error   t-value    P-value    df
Y~X1 0.1550938  0.2205078 0.7033483 0.49136012 17.00
Y~X2 0.4581556  0.2196783 2.0855754 0.05240092 17.00
Y~~Y 0.6538707  0.2242758 2.9154759         NA  4.25
Y    0.7967775  0.2506765 3.1785096 0.00549580 17.00
\end{verbatim}

\subsection{How it works in a nutshell}
\label{sec:orgb76b07a}

When using \textbf{lava}, the p.values that are obtained from the summary
(Wald tests) rely on a Gaussian approximation. While being
asymptotically valid, this approximation may not provide a very
accurate control of the type 1 error rate in small
samples. Simulations have shown that the type 1 error rate tends to be
too large, i.e. the p.values are have a downward bias.  gg
\textbf{lavaSearch2} improves the Gaussian approximation in two way:
\begin{itemize}
\item using a Student's t distribution instead of a normal distribution to
account for the uncertainty on the variance of the coefficients. The
degrees of freedom are estimated using Satterwaite approximation,
i.e. identifying the chi-squared distribution that best fit the
observed moments of the variance of the coefficients.
\item correct for the first order bias in the residual variance due to ML
estimation. This bias also affects the standard error of the
estimates and the control of the type 1 error. The correction does
not change the estimates (i.e. the column "Estimate" in the summary
remain unchanged), but it changes the corresponding standard error
and degree of freedoms (i.e. columns "Std. Error" and "df" in the
summary are modified).
\end{itemize}

\subsection{Single univariate Wald test}
\label{sec:org585b66b}

We will illustrate the functionalities using a simulated dataset:
\lstset{language=r,label= ,caption= ,captionpos=b,numbers=none}
\begin{lstlisting}
## simulate data
mSim <- lvm(Y1~eta,Y2~eta,Y3~0.4+0.4*eta,Y4~0.6+0.6*eta,eta~0.5*X1+0.7*X2)
latent(mSim) <- ~eta
set.seed(12)
df.data <- sim(mSim, n = 3e1, latent = FALSE)

## display
head(df.data)
\end{lstlisting}

\begin{verbatim}
          Y1         Y2          Y3         Y4         X1         X2
1 -1.7606233  0.1264910  0.66442611  0.2579355  0.2523400 -1.5431527
2  3.0459417  2.4631929  0.00283511  2.1714802  0.6423143 -1.3206009
3 -2.1443162 -0.3318033  0.82253070  0.3008415 -0.3469361 -0.6758215
4 -2.5050328 -1.3878987 -0.10474850 -1.7814956 -0.5152632 -0.3670054
5 -2.5307249  0.3012422  1.22046986 -1.0195188  0.3981689 -0.5138722
6 -0.9521366  0.1669496 -0.21422548  1.5954456  0.9535572 -0.9592540
\end{verbatim}

We first fit the latent variable model using, as usual, the \texttt{estimate}
function:
\lstset{language=r,label= ,caption= ,captionpos=b,numbers=none}
\begin{lstlisting}
m <- lvm(c(Y1,Y2,Y3,Y4)~eta, eta~X1+X2)
e <- estimate(m, data = df.data)
\end{lstlisting}

We can extract the Wald tests based on a normal approximation using
\texttt{summary}:
\lstset{language=r,label= ,caption= ,captionpos=b,numbers=none}
\begin{lstlisting}
summary(e)$coef[c("Y2","Y3","Y2~eta","Y3~eta","eta~X1","eta~X2"), ]
\end{lstlisting}

\begin{verbatim}
        Estimate Std. Error   Z-value      P-value
Y2     0.2335412  0.2448593 0.9537775 0.3401962906
Y3     0.5114275  0.1785886 2.8637186 0.0041869974
Y2~eta 0.9192847  0.2621248 3.5070497 0.0004531045
Y3~eta 0.2626930  0.1558978 1.6850339 0.0919820326
eta~X1 0.5150072  0.2513393 2.0490515 0.0404570768
eta~X2 0.6212222  0.2118930 2.9317729 0.0033703310
\end{verbatim}

As explain at the begining of this section, \textbf{lavaSearch2} implements
two corrections that can be directly applied by calling the \texttt{summary2}
method:
\lstset{language=r,label= ,caption= ,captionpos=b,numbers=none}
\begin{lstlisting}
summary2(e)$coef[c("Y2","Y3","Y2~eta","Y3~eta","eta~X1","eta~X2"), ]
\end{lstlisting}

\begin{verbatim}
        Estimate Std. Error   t-value     P-value        df
Y2     0.2335412  0.2518218 0.9274067 0.371516094 12.328385
Y3     0.5114275  0.1828716 2.7966475 0.009848769 24.707696
Y2~eta 0.9192847  0.2653220 3.4647887 0.031585600  3.515034
Y3~eta 0.2626930  0.1562776 1.6809386 0.143826633  5.993407
eta~X1 0.5150072  0.2642257 1.9491180 0.065414617 20.044312
eta~X2 0.6212222  0.2221293 2.7966698 0.009275494 27.718363
\end{verbatim}

To use the Satterthwaite correction alone, set the argument
  \texttt{bias.correct} to \texttt{FALSE}:

\lstset{language=r,label= ,caption= ,captionpos=b,numbers=none}
\begin{lstlisting}
summary2(e, bias.correct = FALSE)$coef[c("Y2","Y3","Y2~eta","Y3~eta","eta~X1","eta~X2"), ]
\end{lstlisting}

\begin{verbatim}
        Estimate Std. Error   t-value     P-value        df
Y2     0.2335412  0.2448593 0.9537775 0.357711941 12.911877
Y3     0.5114275  0.1785886 2.8637186 0.008210968 25.780552
Y2~eta 0.9192847  0.2621248 3.5070497 0.028396459  3.674640
Y3~eta 0.2626930  0.1558978 1.6850339 0.141185621  6.222912
eta~X1 0.5150072  0.2513393 2.0490515 0.052814794 21.571210
eta~X2 0.6212222  0.2118930 2.9317729 0.006351686 30.370334
\end{verbatim}

When using the Satterthwaite correction alone, the standard error are
left unchanged compared to the original lava output. The only change
is how the p-values are computed, i.e. based on the quantiles of a
Student's t distribution instead of a Gaussian distribution. 

\subsection{Saving computation time with \texttt{sCorrect}}
\label{sec:orgbd325f9}
For each call to \texttt{summary2} the small sample size correction(s) will
be recalculated. However the calculation of the sample correction(s)
can be time consuming.
\lstset{language=r,label= ,caption= ,captionpos=b,numbers=none}
\begin{lstlisting}
system.time(
	res <- summary2(e, bias.correct = FALSE)
)
\end{lstlisting}

\begin{verbatim}
user  system elapsed 
0.24    0.00    0.24
\end{verbatim}

In such a case one can pre-compute the main terms of the correction
(e.g. the derivative of the variance-covariance matrix) once for all
using the \texttt{sCorrect} method (\texttt{sCorrect} stands for Satterthwaite
correction). When calling \texttt{sCorrect}, the right hand side indicates
whether the bias correction should be used (equivalent to
\texttt{bias.correct} argument described previously):
\lstset{language=r,label= ,caption= ,captionpos=b,numbers=none}
\begin{lstlisting}
e2 <- e
sCorrect(e2) <- TRUE
\end{lstlisting}

\texttt{sCorrect} automatically store the pre-computed terms in the \texttt{sCorrect}
slot of the object. It also adds the class \texttt{lvmfit2} to the object:
\lstset{language=r,label= ,caption= ,captionpos=b,numbers=none}
\begin{lstlisting}
class(e2)
\end{lstlisting}
\begin{verbatim}
[1] "lvmfit2" "lvmfit"
\end{verbatim}

Then p-values computed using the small sample correction can be
obtained calling the \texttt{summary} method, as usual:
\lstset{language=r,label= ,caption= ,captionpos=b,numbers=none}
\begin{lstlisting}
summary2(e2)$coef[c("Y2","Y3","Y2~eta","Y3~eta","eta~X1","eta~X2"), ]
\end{lstlisting}

\begin{verbatim}
        Estimate Std. Error   t-value     P-value        df
Y2     0.2335412  0.2518218 0.9274067 0.371516094 12.328385
Y3     0.5114275  0.1828716 2.7966475 0.009848769 24.707696
Y2~eta 0.9192847  0.2653220 3.4647887 0.031585600  3.515034
Y3~eta 0.2626930  0.1562776 1.6809386 0.143826633  5.993407
eta~X1 0.5150072  0.2642257 1.9491180 0.065414617 20.044312
eta~X2 0.6212222  0.2221293 2.7966698 0.009275494 27.718363
\end{verbatim}

The \texttt{summary2} methods take approximately the same time as the usual
\texttt{summary} method:
\lstset{language=r,label= ,caption= ,captionpos=b,numbers=none}
\begin{lstlisting}
system.time(
	summary2(e2)
)
\end{lstlisting}

\begin{verbatim}
user  system elapsed 
0.10    0.00    0.09
\end{verbatim}

\lstset{language=r,label= ,caption= ,captionpos=b,numbers=none}
\begin{lstlisting}
system.time(
	summary(e2)
)
\end{lstlisting}

\begin{verbatim}
user  system elapsed 
0.08    0.00    0.08
\end{verbatim}

\subsection{Single multivariate Wald test}
\label{sec:orgf03d592}

The function \texttt{compare} can be use to perform multivariate Wald tests,
i.e. to test simultaneously several linear combinations of the
coefficients.  \texttt{compare} uses a contrast matrix to encode in lines
which linear combination of coefficients should be tested. For
instance if we want to simultaneously test whether all the mean
coefficients are 0, we can create a contrast matrix using
\texttt{createContrast}:
\lstset{language=r,label= ,caption= ,captionpos=b,numbers=none}
\begin{lstlisting}
resC <- createContrast(e2, par = c("Y2=0","Y2~eta=0","eta~X1=0"))
resC
\end{lstlisting}

\begin{verbatim}
$contrast
             Y2 Y3 Y4 eta Y2~eta Y3~eta Y4~eta eta~X1 eta~X2 Y1~~Y1 Y2~~Y2 Y3~~Y3 Y4~~Y4
[Y2] = 0      1  0  0   0      0      0      0      0      0      0      0      0      0
[Y2~eta] = 0  0  0  0   0      1      0      0      0      0      0      0      0      0
[eta~X1] = 0  0  0  0   0      0      0      0      1      0      0      0      0      0
             eta~~eta
[Y2] = 0            0
[Y2~eta] = 0        0
[eta~X1] = 0        0

$null
    [Y2] = 0 [Y2~eta] = 0 [eta~X1] = 0 
           0            0            0 

$Q
[1] 3
\end{verbatim}

We can then test the linear hypothesis by specifying in \texttt{compare} the
left hand side of the hypothesis (argument contrast) and the right
hand side (argument null):
\lstset{language=r,label= ,caption= ,captionpos=b,numbers=none}
\begin{lstlisting}
resTest0 <- lava::compare(e2, contrast = resC$contrast, null = resC$null)
resTest0
\end{lstlisting}

\begin{verbatim}
	- Wald test -

	Null Hypothesis:
	[Y2] = 0
	[Y2~eta] = 0
	[eta~X1] = 0

data:  
chisq = 21.332, df = 3, p-value = 8.981e-05
sample estimates:
          Estimate   Std.Err       2.5%     97.5%
[Y2]     0.2335412 0.2448593 -0.2463741 0.7134566
[Y2~eta] 0.9192847 0.2621248  0.4055295 1.4330399
[eta~X1] 0.5150072 0.2513393  0.0223912 1.0076231
\end{verbatim}

\texttt{compare} uses a chi-squared distribution to compute the p-values.
Similarly to the Gaussian approximation, while being valid
asymptotically this procedure may not provide a very accurate control
of the type 1 error rate in small samples. Fortunately, the correction
proposed for the univariate Wald statistic can be adapted to the
multivariate Wald statistic. This is achieved by \texttt{compare2}:
\lstset{language=r,label= ,caption= ,captionpos=b,numbers=none}
\begin{lstlisting}
resTest1 <- compare2(e2, contrast = resC$contrast, null = resC$null)
resTest1
\end{lstlisting}

\begin{verbatim}
	- Wald test -

	Null Hypothesis:
	[Y2] = 0
	[Y2~eta] = 0
	[eta~X1] = 0

data:  
F-statistic = 6.7118, df1 = 3, df2 = 11.1, p-value = 0.007596
sample estimates:
              Estimate   Std.Err        df       2.5%     97.5%
[Y2] = 0     0.2335412 0.2518218 12.328385 -0.3135148 0.7805973
[Y2~eta] = 0 0.9192847 0.2653220  3.515034  0.1407653 1.6978041
[eta~X1] = 0 0.5150072 0.2642257 20.044312 -0.0360800 1.0660943
\end{verbatim}

The same result could have been obtained using the par argument to
define the linear hypothesis:
\lstset{language=r,label= ,caption= ,captionpos=b,numbers=none}
\begin{lstlisting}
resTest2 <- compare2(e2, par = c("Y2","Y2~eta","eta~X1"))
identical(resTest1,resTest2)
\end{lstlisting}

\begin{verbatim}
[1] TRUE
\end{verbatim}

Now a F distribution is used to compute the p-values. As before on can
set the argument \texttt{bias.correct} to \texttt{FALSE} to use the Satterthwaite
approximation alone:
\lstset{language=r,label= ,caption= ,captionpos=b,numbers=none}
\begin{lstlisting}
resTest3 <- compare2(e, bias.correct = FALSE, 
					  contrast = resC$contrast, null = resC$null)
resTest3
\end{lstlisting}

\begin{verbatim}
	- Wald test -

	Null Hypothesis:
	[Y2] = 0
	[Y2~eta] = 0
	[eta~X1] = 0

data:  
F-statistic = 7.1107, df1 = 3, df2 = 11.13, p-value = 0.006182
sample estimates:
              Estimate   Std.Err       df         2.5%     97.5%
[Y2] = 0     0.2335412 0.2448593 12.91188 -0.295812256 0.7628948
[Y2~eta] = 0 0.9192847 0.2621248  3.67464  0.165378080 1.6731913
[eta~X1] = 0 0.5150072 0.2513393 21.57121 -0.006840023 1.0368543
\end{verbatim}

In this case the F-statistic of \texttt{compare2} is the same as the
chi-squared statistic of \texttt{compare} divided by the rank of the contrast matrix:
\lstset{language=r,label= ,caption= ,captionpos=b,numbers=none}
\begin{lstlisting}
resTest0$statistic/qr(resC$contrast)$rank
\end{lstlisting}

\begin{verbatim}
   chisq 
7.110689
\end{verbatim}

\subsection{Robust Wald tests}
\label{sec:orgd7296ae}

When one does not want to assume normality distributed residuals,
robust standard error can be used instead of the model based standard
errors. They can be obtain by setting the argument \texttt{robust} to \texttt{TRUE}
when computing univariate Wald tests:
\lstset{language=r,label= ,caption= ,captionpos=b,numbers=none}
\begin{lstlisting}
summary2(e, robust = TRUE)$coef[c("Y2","Y3","Y2~eta","Y3~eta","eta~X1","eta~X2"), ]
\end{lstlisting}

\begin{verbatim}
        Estimate robust SE   t-value     P-value        df
Y2     0.2335412 0.2353245 0.9924222 0.340071187 12.328385
Y3     0.5114275 0.1897160 2.6957535 0.012449993 24.707696
Y2~eta 0.9192847 0.1791240 5.1321150 0.009609880  3.515034
Y3~eta 0.2626930 0.1365520 1.9237585 0.102782655  5.993407
eta~X1 0.5150072 0.2167580 2.3759546 0.027585480 20.044312
eta~X2 0.6212222 0.2036501 3.0504389 0.004989038 27.718363
\end{verbatim}

or multivariate Wald test:
\lstset{language=r,label= ,caption= ,captionpos=b,numbers=none}
\begin{lstlisting}
compare2(e2, robust = TRUE, par = c("Y2","Y2~eta","eta~X1"))
\end{lstlisting}

\begin{verbatim}
	- Wald test -

	Null Hypothesis:
	[Y2] = 0
	[Y2~eta] = 0
	[eta~X1] = 0

data:  
F-statistic = 12.526, df1 = 3, df2 = 11.1, p-value = 0.0006959
sample estimates:
              Estimate robust SE        df        2.5%     97.5%
[Y2] = 0     0.2335412 0.2353245 12.328385 -0.27767612 0.7447586
[Y2~eta] = 0 0.9192847 0.1791240  3.515034  0.39369139 1.4448780
[eta~X1] = 0 0.5150072 0.2167580 20.044312  0.06292197 0.9670923
\end{verbatim}

Only the standard error is affected by the argument \texttt{robust}, the
degrees of freedom are the one of the model-based standard errors.  It
may be surprising that the (corrected) robust standard errors are (in
this example) smaller than the (corrected) model-based one. This is
also the case for the uncorrected one:
\lstset{language=r,label= ,caption= ,captionpos=b,numbers=none}
\begin{lstlisting}
rbind(robust = diag(crossprod(iid(e2))),
	  model = diag(vcov(e2)))
\end{lstlisting}

\begin{verbatim}
               Y2         Y3         Y4        eta     Y2~eta     Y3~eta     Y4~eta
robust 0.04777252 0.03325435 0.03886706 0.06011727 0.08590732 0.02179453 0.02981895
model  0.05995606 0.03189389 0.04644303 0.06132384 0.06870941 0.02430412 0.03715633
           eta~X1     eta~X2    Y1~~Y1    Y2~~Y2     Y3~~Y3     Y4~~Y4  eta~~eta
robust 0.05166005 0.05709393 0.2795272 0.1078948 0.03769614 0.06923165 0.3198022
model  0.06317144 0.04489865 0.1754744 0.1600112 0.05112998 0.10152642 0.2320190
\end{verbatim}

This may be explained by the fact the robust standard error tends to
be liberal in small samples (e.g. see Kauermann 2001, A Note on the
Efficiency of Sandwich Covariance Matrix Estimation ).
\subsection{Assessing the type 1 error of the testing procedure}
\label{sec:orgfe15dec}

The function \texttt{calibrateType1} can be used to assess the type 1 error
of a Wald statistic on a specific example. This however assumes that
the estimated model is correctly specified. Let's make an example. For
this we simulate some data:
\lstset{language=r,label= ,caption= ,captionpos=b,numbers=none}
\begin{lstlisting}
set.seed(10)
m.generative <- lvm(Y ~ X1 + X2 + Gene)
categorical(m.generative, labels = c("ss","ll")) <- ~Gene
d <- lava::sim(m.generative, n = 50, latent = FALSE)
\end{lstlisting}

Let's now imagine that we want to analyze the relationship between
Y and Gene using the following dataset:
\lstset{language=r,label= ,caption= ,captionpos=b,numbers=none}
\begin{lstlisting}
head(d)
\end{lstlisting}

\begin{verbatim}
            Y         X1         X2 Gene
1 -1.14369572 -0.4006375 -0.7618043   ss
2 -0.09943370 -0.3345566  0.4193754   ss
3 -0.04331996  1.3679540 -1.0399434   ll
4  2.25017335  2.1377671  0.7115740   ss
5  0.16715138  0.5058193 -0.6332130   ss
6  1.73931135  0.7863424  0.5631747   ss
\end{verbatim}

For this we fit define a LVM:
\lstset{language=r,label= ,caption= ,captionpos=b,numbers=none}
\begin{lstlisting}
myModel <- lvm(Y ~ X1 + X2 + Gene)
\end{lstlisting}

and estimate the coefficients of the model using \texttt{estimate}:
\lstset{language=r,label= ,caption= ,captionpos=b,numbers=none}
\begin{lstlisting}
e <- estimate(myModel, data = d)
e
\end{lstlisting}
\begin{verbatim}
                    Estimate Std. Error  Z-value  P-value
Regressions:                                             
   Y~X1              1.02349    0.12017  8.51728   <1e-12
   Y~X2              0.91519    0.12380  7.39244   <1e-12
   Y~Genell          0.48035    0.23991  2.00224  0.04526
Intercepts:                                              
   Y                -0.11221    0.15773 -0.71141   0.4768
Residual Variances:                                      
   Y                 0.67073    0.13415  5.00000
\end{verbatim}

We can now use \texttt{calibrateType1} to perform a simulation study. We just
need to define the null hypotheses (i.e. which coefficients should be
set to 0 when generating the data) and the number of simulations:
\lstset{language=r,label= ,caption= ,captionpos=b,numbers=none}
\begin{lstlisting}
mySimulation <- calibrateType1(e, 
							   null = "Y~Genell",
							   n.rep = 50, 
							   trace = FALSE, seed = 10)
\end{lstlisting}

To save time we only make 50 simulations but much more are necessary
to really assess the type 1 error rate. Then we can use the \texttt{summary}
method to display the results:
\lstset{language=r,label= ,caption= ,captionpos=b,numbers=none}
\begin{lstlisting}
summary(mySimulation)
\end{lstlisting}

\begin{verbatim}
Estimated type 1 error rate [95% confidence interval] 
  > sample size: 50 | number of simulations: 50
     link statistic correction type1error                  CI
 Y~Genell      Wald       Gaus       0.12 [0.05492 ; 0.24242]
                          Satt       0.10 [0.04224 ; 0.21869]
                           SSC       0.10 [0.04224 ; 0.21869]
                    SSC + Satt       0.08 [0.03035 ; 0.19456]

Corrections: Gaus = Gaussian approximation 
             SSC  = small sample correction 
             Satt = Satterthwaite approximation
\end{verbatim}


\clearpage

\section{Adjustment for multiple comparisons}
\label{sec:org6609fde}
\subsection{Univariate Wald test, single model}
\label{sec:org5609a6f}

When performing multiple testing, adjustment for multiple comparisons
is necessary in order to control the type 1 error rate, i.e. to
provide interpretable p.values. The \textbf{multcomp} package enables to do
such adjustment when all tests comes from the same \texttt{lvmfit} object:
\lstset{language=r,label= ,caption= ,captionpos=b,numbers=none}
\begin{lstlisting}
## simulate data
mSim <- lvm(Y ~ 0.25 * X1 + 0.3 * X2 + 0.35 * X3 + 0.4 * X4 + 0.45 * X5 + 0.5 * X6)
set.seed(10)
df.data <- sim(mSim, n = 4e1)

## fit lvm
e.lvm <- estimate(lvm(Y ~ X1 + X2 + X3 + X4 + X5 + X6), data = df.data)
name.coef <- names(coef(e.lvm))
n.coef <- length(name.coef)

## Create contrast matrix
resC <- createContrast(e.lvm, par = paste0("Y~X",1:6), rowname.rhs = FALSE)
resC$contrast
\end{lstlisting}

\begin{verbatim}
     Y Y~X1 Y~X2 Y~X3 Y~X4 Y~X5 Y~X6 Y~~Y
Y~X1 0    1    0    0    0    0    0    0
Y~X2 0    0    1    0    0    0    0    0
Y~X3 0    0    0    1    0    0    0    0
Y~X4 0    0    0    0    1    0    0    0
Y~X5 0    0    0    0    0    1    0    0
Y~X6 0    0    0    0    0    0    1    0
\end{verbatim}

\lstset{language=r,label= ,caption= ,captionpos=b,numbers=none}
\begin{lstlisting}
e.glht <- multcomp::glht(e.lvm, linfct = resC$contrast, rhs = resC$null)
summary(e.glht)
\end{lstlisting}
\begin{verbatim}
	 Simultaneous Tests for General Linear Hypotheses

Fit: estimate.lvm(x = lvm(Y ~ X1 + X2 + X3 + X4 + X5 + X6), data = df.data)

Linear Hypotheses:
          Estimate Std. Error z value Pr(>|z|)   
Y~X1 == 0   0.3270     0.1589   2.058  0.20725   
Y~X2 == 0   0.4025     0.1596   2.523  0.06611 . 
Y~X3 == 0   0.5072     0.1383   3.669  0.00144 **
Y~X4 == 0   0.3161     0.1662   1.902  0.28582   
Y~X5 == 0   0.3875     0.1498   2.586  0.05554 . 
Y~X6 == 0   0.3758     0.1314   2.859  0.02482 * 
---
Signif. codes:  0 '***' 0.001 '**' 0.01 '*' 0.05 '.' 0.1 ' ' 1
(Adjusted p values reported -- single-step method)
\end{verbatim}

Note that this correction relies on the Gaussian approximation. To use
small sample corrections implemented in \textbf{lavaSearch2}, just call
\texttt{glht2} instead of \texttt{glht}:
\lstset{language=r,label= ,caption= ,captionpos=b,numbers=none}
\begin{lstlisting}
e.glht2 <- glht2(e.lvm, linfct = resC$contrast, rhs = resC$null)
summary(e.glht2)
\end{lstlisting}

\begin{verbatim}
	 Simultaneous Tests for General Linear Hypotheses

Fit: estimate.lvm(x = lvm(Y ~ X1 + X2 + X3 + X4 + X5 + X6), data = df.data)

Linear Hypotheses:
          Estimate Std. Error t value Pr(>|t|)  
Y~X1 == 0   0.3270     0.1750   1.869   0.3290  
Y~X2 == 0   0.4025     0.1757   2.291   0.1482  
Y~X3 == 0   0.5072     0.1522   3.333   0.0123 *
Y~X4 == 0   0.3161     0.1830   1.727   0.4128  
Y~X5 == 0   0.3875     0.1650   2.349   0.1315  
Y~X6 == 0   0.3758     0.1447   2.597   0.0762 .
---
Signif. codes:  0 '***' 0.001 '**' 0.01 '*' 0.05 '.' 0.1 ' ' 1
(Adjusted p values reported -- single-step method)
\end{verbatim}

The single step method is the appropriate correction when one wants to
report the most significant p-value relative to a set of
hypotheses. If the second most significant p-value is also to be
reported then the method "free" is more appropriate:
\lstset{language=r,label= ,caption= ,captionpos=b,numbers=none}
\begin{lstlisting}
summary(e.glht2, test = multcomp::adjusted("free"))
\end{lstlisting}

\begin{verbatim}
	 Simultaneous Tests for General Linear Hypotheses

Fit: estimate.lvm(x = lvm(Y ~ X1 + X2 + X3 + X4 + X5 + X6), data = df.data)

Linear Hypotheses:
          Estimate Std. Error t value Pr(>|t|)  
Y~X1 == 0   0.3270     0.1750   1.869   0.1291  
Y~X2 == 0   0.4025     0.1757   2.291   0.0913 .
Y~X3 == 0   0.5072     0.1522   3.333   0.0123 *
Y~X4 == 0   0.3161     0.1830   1.727   0.1291  
Y~X5 == 0   0.3875     0.1650   2.349   0.0913 .
Y~X6 == 0   0.3758     0.1447   2.597   0.0645 .
---
Signif. codes:  0 '***' 0.001 '**' 0.01 '*' 0.05 '.' 0.1 ' ' 1
(Adjusted p values reported -- free method)
\end{verbatim}
Indeed, here there is no relations between the hypotheses. See the
book: "Multiple Comparisons Using R" by Frank Bretz, Torsten Hothorn,
and Peter Westfall (2011, CRC Press) for details about the theory
underlying the \textbf{multcomp} package.

\subsection{Univariate Wald test, multiple models}
\label{sec:orga0473a2}

Pipper et al. in "A Versatile Method for Confirmatory Evaluation of
the Effects of a Covariate in Multiple Models" (2012, Journal of the
Royal Statistical Society, Series C) developed a method to assess the
effect of an exposure on several outcomes when a different model is
fitted for each outcome. This method has been implemented in the \texttt{mmm}
function from the \textbf{multcomp} package for glm and Cox
models. \textbf{lavaSearch2} extends it to \texttt{lvm}. 

Let's consider an example where we wish to assess the treatment effect
on three outcomes X, Y, and Z. We have at hand three measurements
relative to outcome Z for each individual:
\lstset{language=r,label= ,caption= ,captionpos=b,numbers=none}
\begin{lstlisting}
mSim <- lvm(X ~ Age + 0.5*Treatment,
			Y ~ Gender + 0.25*Treatment,
			c(Z1,Z2,Z3) ~ eta, eta ~ 0.75*treatment,
			Age[40:5]~1)
latent(mSim) <- ~eta
categorical(mSim, labels = c("placebo","SSRI")) <- ~Treatment
categorical(mSim, labels = c("male","female")) <- ~Gender

n <- 5e1
set.seed(10)
df.data <- sim(mSim, n = n, latent = FALSE)
head(df.data)
\end{lstlisting}

\begin{verbatim}
         X      Age Treatment          Y Gender         Z1         Z2          Z3
1 39.12289 39.10415   placebo  0.6088958 female  1.8714112  2.2960633 -0.09326935
2 39.56766 39.25191      SSRI  1.0001325 female  0.9709943  0.6296226  1.31035910
3 41.68751 43.05884   placebo  2.1551047 female -1.1634011 -0.3332927 -1.30769267
4 44.68102 44.78019      SSRI  0.3852728 female -1.0305476  0.6678775  0.99780139
5 41.42559 41.13105   placebo -0.8666783   male -1.6342816 -0.8285492  1.20450488
6 42.64811 41.75832      SSRI -1.0710170 female -1.2198019 -1.9602130 -1.85472132
   treatment
1  1.1639675
2 -1.5233846
3 -2.5183351
4 -0.7075292
5 -0.2874329
6 -0.4353083
\end{verbatim}

We fit a model specific to each outcome:
\lstset{language=r,label= ,caption= ,captionpos=b,numbers=none}
\begin{lstlisting}
lmX <- lm(X ~ Age + Treatment, data = df.data)
lvmY <- estimate(lvm(Y ~ Gender + Treatment), data = df.data)
lvmZ <- estimate(lvm(c(Z1,Z2,Z3) ~ 1*eta, eta ~ -1 + Treatment), 
				 data = df.data)
\end{lstlisting}

and combine them into a list of \texttt{lvmfit} objects:
\lstset{language=r,label= ,caption= ,captionpos=b,numbers=none}
\begin{lstlisting}
mmm.lvm <- multcomp::mmm(X = lmX, Y = lvmY, Z = lvmZ)
\end{lstlisting}

We can then generate a contrast matrix to test each coefficient
related to the treatment:
\lstset{language=r,label= ,caption= ,captionpos=b,numbers=none}
\begin{lstlisting}
resC <- createContrast(mmm.lvm, var.test = "Treatment", add.variance = TRUE)
resC$contrast
\end{lstlisting}

\begin{verbatim}
                     X: (Intercept) X: Age X: TreatmentSSRI X: sigma2 Y: Y
X: TreatmentSSRI                  0      0                1         0    0
Y: Y~TreatmentSSRI                0      0                0         0    0
Z: eta~TreatmentSSRI              0      0                0         0    0
                     Y: Y~Genderfemale Y: Y~TreatmentSSRI Y: Y~~Y Z: Z1 Z: Z2 Z: Z3
X: TreatmentSSRI                     0                  0       0     0     0     0
Y: Y~TreatmentSSRI                   0                  1       0     0     0     0
Z: eta~TreatmentSSRI                 0                  0       0     0     0     0
                     Z: eta~TreatmentSSRI Z: Z1~~Z1 Z: Z2~~Z2 Z: Z3~~Z3 Z: eta~~eta
X: TreatmentSSRI                        0         0         0         0           0
Y: Y~TreatmentSSRI                      0         0         0         0           0
Z: eta~TreatmentSSRI                    1         0         0         0           0
\end{verbatim}

\lstset{language=r,label= ,caption= ,captionpos=b,numbers=none}
\begin{lstlisting}
lvm.glht2 <- glht2(mmm.lvm, linfct = resC$contrast, rhs = resC$null)
summary(lvm.glht2)
\end{lstlisting}

\begin{verbatim}

	 Simultaneous Tests for General Linear Hypotheses

Linear Hypotheses:
                          Estimate Std. Error t value Pr(>|t|)
X: TreatmentSSRI == 0       0.4661     0.2533   1.840    0.187
Y: Y~TreatmentSSRI == 0    -0.5421     0.2613  -2.074    0.117
Z: eta~TreatmentSSRI == 0  -0.6198     0.4404  -1.407    0.393
(Adjusted p values reported -- single-step method)
\end{verbatim}

This can be compared to the unadjusted p.values:
\lstset{language=r,label= ,caption= ,captionpos=b,numbers=none}
\begin{lstlisting}
summary(lvm.glht2, test = multcomp::univariate())
\end{lstlisting}

\begin{verbatim}
	 Simultaneous Tests for General Linear Hypotheses

Linear Hypotheses:
                          Estimate Std. Error t value Pr(>|t|)  
X: TreatmentSSRI == 0       0.4661     0.2533   1.840   0.0720 .
Y: Y~TreatmentSSRI == 0    -0.5421     0.2613  -2.074   0.0435 *
Z: eta~TreatmentSSRI == 0  -0.6198     0.4404  -1.407   0.1659  
---
Signif. codes:  0 '***' 0.001 '**' 0.01 '*' 0.05 '.' 0.1 ' ' 1
(Univariate p values reported)
\end{verbatim}


\clearpage 
\section{Model diagnostic}
\label{sec:org41ca886}
\subsection{Detection of local dependencies}
\label{sec:org8e05d8d}

The \texttt{modelsearch} function of \textbf{lava} is a diagnostic tool for latent
variable models. It enables to search for local dependencies
(i.e. model misspecification) and add them to the model. Obviously it
is a data-driven procedure and its usefulness can be discussed,
especially in small samples:
\begin{itemize}
\item the procedure is instable, i.e. is likely to lead to two different
models when applied on two different dataset sampled from the same
generative model.
\item it is hard to define a meaningful significance threshold since
p-values should be adjusted for multiple comparisons and sequential
testing. However traditional methods like Bonferonni-Holm tend to
over corrected and therefore reduce the power of the procedure since
they assume that the test are independent.
\end{itemize}

The function \texttt{modelsearch2} in \textbf{lavaSearch2} partially solves the
second issue by adjusting the p-values for multiple testing.

Let's see an example:
\lstset{language=r,label= ,caption= ,captionpos=b,numbers=none}
\begin{lstlisting}
## simulate data
mSim <- lvm(c(y1,y2,y3)~u, u~x1+x2)
latent(mSim) <- ~u
covariance(mSim) <- y2~y3
transform(mSim, Id~u) <- function(x){1:NROW(x)}
set.seed(10)
df.data <- lava::sim(mSim, n = 125, latent = FALSE)
head(df.data)
\end{lstlisting}

\begin{verbatim}
          y1           y2         y3         x1         x2 Id
1  5.5071523  4.883752014  6.2928016  0.8694750  2.3991549  1
2 -0.6398644  0.025832617  0.5088030 -0.6800096 -0.0898721  2
3 -2.5835495 -2.616715027 -2.8982645  0.1732145 -0.8216484  3
4 -2.5312637 -2.518185427 -2.9015033 -0.1594380 -0.2869618  4
5  1.6346220 -0.001877577  0.3705181  0.7934994  0.1312789  5
6  0.4939972  1.759884014  1.5010499  1.6943505 -1.0620840  6
\end{verbatim}

\lstset{language=r,label= ,caption= ,captionpos=b,numbers=none}
\begin{lstlisting}
## fit model
m <- lvm(c(y1,y2,y3)~u, u~x1)
latent(m) <- ~u
addvar(m) <- ~x2 
e.lvm <- estimate(m, data = df.data)
\end{lstlisting}

\texttt{modelsearch2} can be used to sequentially apply the \texttt{modelsearch}
function with a given correction for the p.values:
\lstset{language=r,label= ,caption= ,captionpos=b,numbers=none}
\begin{lstlisting}
resScore <- modelsearch2(e.lvm, alpha = 0.1, trace = FALSE)
displayScore <- summary(resScore)
\end{lstlisting}

\begin{verbatim}
Sequential search for local dependence using the score statistic 
The variable selection procedure retained 2 variables:
    link statistic      p.value adjusted.p.value dp.Info selected nTests
1   u~x2  6.036264 1.577228e-09     5.008615e-08       1     TRUE     10
2 y2~~y3  2.629176 8.559198e-03     6.055947e-02       1     TRUE      9
3  y3~x1  1.770997 7.656118e-02     2.814424e-01       1    FALSE      8
Confidence level: 0.9 (two sided, adjustement: fastmax)
\end{verbatim}

This indeed matches the highest score statistic found by
\texttt{modelsearch}:
\lstset{language=r,label= ,caption= ,captionpos=b,numbers=none}
\begin{lstlisting}
resScore0 <- modelsearch(e.lvm, silent = TRUE)
c(statistic = sqrt(max(resScore0$test[,"Test Statistic"])), 
  p.value = min(resScore0$test[,"P-value"]))
\end{lstlisting}

\begin{verbatim}
   statistic      p.value 
6.036264e+00 1.577228e-09
\end{verbatim}

We can compare the adjustment using the max distribution to bonferroni:
\lstset{language=r,label= ,caption= ,captionpos=b,numbers=none}
\begin{lstlisting}
data.frame(link = displayScore$table[,"link"],
		   none = displayScore$table[,"p.value"],
		   bonferroni = displayScore$table[,"p.value"]*displayScore$table[1,"nTests"],
		   max = displayScore$table[,"adjusted.p.value"])
\end{lstlisting}

\begin{verbatim}
    link         none   bonferroni          max
1   u~x2 1.577228e-09 1.577228e-08 5.008615e-08
2 y2~~y3 8.559198e-03 8.559198e-02 6.055947e-02
3  y3~x1 7.656118e-02 7.656118e-01 2.814424e-01
\end{verbatim}

\subsection{Checking that the names of the variables in the model match those of the data}
\label{sec:org01de8d5}

When estimating latent variable models using \textbf{lava}, it sometimes
happens that the model does not converge:
\lstset{language=r,label= ,caption= ,captionpos=b,numbers=none}
\begin{lstlisting}
## simulate data
set.seed(10)
df.data <- sim(lvm(Y~X1+X2), 1e2)

## fit model
mWrong <- lvm(Y ~ X + X2)
eWrong <- estimate(mWrong, data = df.data)
\end{lstlisting}

\begin{verbatim}
Warning messages:
1: In estimate.lvm(mWrong, data = df.data) :
  Lack of convergence. Increase number of iteration or change starting values.
2: In sqrt(diag(asVar)) : NaNs produced
\end{verbatim}

This can have several reasons:
\begin{itemize}
\item the model is not identifiable.
\item the optimization routine did not managed to find a local
optimum. This may happen for complex latent variable model where the
objective function is not convex or locally convex.
\item the user has made a mistake when defining the model or has not given
the appropriate dataset.
\end{itemize}

The \texttt{checkData} function enables to check the last point. It compares
the observed variables defined in the model and the one given by the
dataset. In case of mismatch it returns a message:
\lstset{language=r,label= ,caption= ,captionpos=b,numbers=none}
\begin{lstlisting}
checkData(mWrong, df.data)
\end{lstlisting}

\begin{verbatim}
Missing variable in data: X
\end{verbatim}

In presence of latent variables, the user needs to explicitely define
them in the model, otherwise \texttt{checkData} will identify them as an
issue:
\lstset{language=r,label= ,caption= ,captionpos=b,numbers=none}
\begin{lstlisting}
## simulate data
set.seed(10)
mSim <- lvm(c(Y1,Y2,Y3)~eta)
latent(mSim) <- ~eta
df.data <- sim(mSim, n = 1e2, latent = FALSE)

## fit model
m <- lvm(c(Y1,Y2,Y3)~eta)
checkData(m, data = df.data)
\end{lstlisting}

\begin{verbatim}
Missing variable in data: eta
\end{verbatim}

\lstset{language=r,label= ,caption= ,captionpos=b,numbers=none}
\begin{lstlisting}
latent(m) <- ~eta
checkData(m, data = df.data)
\end{lstlisting}

\begin{verbatim}
No issue detected
\end{verbatim}


\clearpage

\section{Information about the R session used for this document}
\label{sec:org33bfc7b}

\lstset{language=r,label= ,caption= ,captionpos=b,numbers=none}
\begin{lstlisting}
sessionInfo()
\end{lstlisting}

\begin{verbatim}
R version 3.5.1 (2018-07-02)
Platform: x86_64-w64-mingw32/x64 (64-bit)
Running under: Windows 7 x64 (build 7601) Service Pack 1

Matrix products: default

locale:
[1] LC_COLLATE=Danish_Denmark.1252  LC_CTYPE=Danish_Denmark.1252   
[3] LC_MONETARY=Danish_Denmark.1252 LC_NUMERIC=C                   
[5] LC_TIME=Danish_Denmark.1252    

attached base packages:
[1] stats     graphics  grDevices utils     datasets  methods   base     

other attached packages:
[1] lavaSearch2_1.4 ggplot2_3.0.0   lava_1.6.3     

loaded via a namespace (and not attached):
 [1] Rcpp_0.12.19      pillar_1.3.0      compiler_3.5.1    plyr_1.8.4       
 [5] bindr_0.1.1       tools_3.5.1       tibble_1.4.2      gtable_0.2.0     
 [9] lattice_0.20-35   pkgconfig_2.0.2   rlang_0.2.2       Matrix_1.2-14    
[13] parallel_3.5.1    mvtnorm_1.0-8     bindrcpp_0.2.2    withr_2.1.2      
[17] dplyr_0.7.6       stringr_1.3.1     grid_3.5.1        tidyselect_0.2.4 
[21] glue_1.3.0        R6_2.2.2          survival_2.42-6   multcomp_1.4-8   
[25] TH.data_1.0-9     purrr_0.2.5       reshape2_1.4.3    magrittr_1.5     
[29] scales_1.0.0      codetools_0.2-15  MASS_7.3-50       splines_3.5.1    
[33] assertthat_0.2.0  colorspace_1.3-2  numDeriv_2016.8-1 sandwich_2.5-0   
[37] stringi_1.2.4     lazyeval_0.2.1    munsell_0.5.0     crayon_1.3.4     
[41] zoo_1.8-4
\end{verbatim}
\end{document}